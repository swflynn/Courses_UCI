\documentclass{article}
%==============================================================================%
%	                          Packages                                     %
%==============================================================================%
% Packages
\usepackage[utf8]{inputenc}
\usepackage{graphicx}
\usepackage{float}
\usepackage{amsmath}
\usepackage{amssymb}
\usepackage{braket}
\usepackage{subcaption}
\usepackage[margin=0.7in]{geometry}
\usepackage[version=4]{mhchem}
%==============================================================================%
%                           User-Defined Commands                              %
%==============================================================================%
% User-Defined Commands
\newcommand{\be}{\begin{equation}}
\newcommand{\ee}{\end{equation}}
\newcommand{\benum}{\begin{enumerate}}
\newcommand{\eenum}{\end{enumerate}}
\newcommand{\pd}{\partial}
\newcommand{\dg}{\dagger}
\newcommand{\sumzero}{\sum_{n=0}^\infty}
\newcommand{\sumone}{\sum_{n=1}^\infty}
\newcommand{\bA}{\mbox{\bf A}}
%==============================================================================%
%                             Title Information                                %
%==============================================================================%
\title{Chem237: Lecture 13}
\date{5/3/18}
\author{Shane Flynn}
%==============================================================================%
%	Everyone Please Make Comments if Something Needs to be Reviewed        %
%                           Or just fix it yourself!                           %
%==============================================================================%
\begin{document}
\maketitle

\section{Note}
9-9-19; Just copied the lecture notes I have over, this has not been reviewed/edited/made ready for the world. 

\section*{Gaussian Elimination}
Reduce your matrix to an upper-triangular (or lower triangular) form by performing operations such that the determinant doesn't change. 
%==============================================================================%
%                               Note/Question:
%(Are we using bold capital letters for matricies? let's be consistent with notation). 
%==============================================================================%

Consider the square N by N matrix $\bA$, which has column vectors labeled $\vec{a}_i$

\be
\bA =  
\begin{bmatrix}
    A_{11}  & \dotsb &  A_{1N} \\
    \vdots  & \ddots &  \vdots \\
    A_{1N}  & \dotsb &  A_{NN} 
\end{bmatrix}
    = 
\begin{bmatrix}
        \vec{a}_1 & \dots  & \vec{a}_N 
\end{bmatrix}
\ee

To compute the determinant of $\bA$ we know the following
\be
\be \det\left(\bA\right) = \det\left(\vec{a}_1 \dots  \vec{a}_N \right) = \det\left(\vec{a}_1- \lambda_{a2}, \vec{a}_2 \dots  \vec{a}_N \right) 
\ee

Let $\lambda_1 = \frac{A_{N1}}{A_{N2}}$ then we have

\end{document}
