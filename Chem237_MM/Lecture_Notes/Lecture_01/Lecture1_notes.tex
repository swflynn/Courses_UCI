\documentclass{article}
%==============================================================================%
%	                          Packages                                     %
%==============================================================================%
% Packages
\usepackage[utf8]{inputenc}
\usepackage{graphicx}
\usepackage{amsmath}
\usepackage{braket}
\usepackage[margin=0.7in]{geometry}
\usepackage[version=4]{mhchem}
%==============================================================================%
%                           User-Defined Commands                              %
%==============================================================================%
% User-Defined Commands
\newcommand{\be}{\begin{equation}}
\newcommand{\ee}{\end{equation}}
\newcommand{\benum}{\begin{enumerate}}
\newcommand{\eenum}{\end{enumerate}}
\newcommand{\pd}{\partial}
\newcommand{\dg}{\dagger}
%==============================================================================%
%                             Title Information                                %
%==============================================================================%
\title{Chem237: Lecture 1}
\date{4/3/18}
\author{Shane Flynn, Alan Robledo}
%==============================================================================%
%	Everyone Please Make Comments if Something Needs to be Reviewed        %
%                           Or just fix it yourself!                           %
%==============================================================================%
\begin{document}
\maketitle
\section*{Course Overview}
The course will not directly follow the textbook, however, it will provide a rough guideline for important topics.
The lecture and homework are the most important aspects of the course.
The homework will be split into both analytical and numerical problems, any programming language is fine for the numerical problems, however, mathematica and other highly developed languages are discouraged.
For the numerical problems if you are asked to make an algorithm you CANNOT use pre-built algorithms available to the languages.

\section*{Chapter 1: Differential Equations}
A Differential Equation simply refers to an equation containing derivatives.
An \textbf{Ordinary Differential Equation} (ODE) refers to any differential equation with derivatives taken wrt a single variable.
In contrast a \textbf{Partial Differential Equation} (PDE) contains derivatives wrt more than 1 variable.
We can further categorize differential equations (both ODE and PDE) into linear or nonlinear.
As you can imagine, linear equations are simpler to solve compared to nonlinear and ODEs are easier to solve than PDEs.

\subsection*{Linear Equations}
Linear Equations are a general topic, there are linear differential equations, linear algebraic equations, and etc.
A linear equation has the form
\be
\hat{A} f(x,y,\hdots) = B(x,y,\hdots)
\ee
Where the RHS (B) is known, and the function f is an unknown function.
To be a linear equation, the operator $\hat{A}$ must be a linear operator.
For example, take the equation above and let $\hat{A}$ be the differential operator wrt a single dimension, and you get a linear ordinary differential equation.

If the RHS (B) is equal to 0, the linear equation is called \textbf{Homogeneous} and if the RHS is anything other than 0, the linear equation is called \textbf{Inhomogeneous}.

\subsubsection*{Homogeneous Linear Equations}
Consider equation \ref{eq:lin_homo} below
\be \label{eq:lin_homo}
\hat{A}f = 0
\ee
If we assume f$_1$ and f$_2$ are solutions to equation \ref{eq:lin_homo}, then we immediately know another solution to be a linear combination of f$_1$ and f$_2$
\be
f_3 = c_1f_1 + c_2f_2 .
\ee
This result is true for any set of linear homogeneous equations.
The linear combination of two unique solutions to a linear homogeneous equation gives another unique solution.

\subsubsection*{Inhomogeneous Linear Equations}
If we now consider the Inhomogeneous linear equation, equation \ref{eq:lin_inhomo}
\be \label{eq:lin_inhomo}
\hat{A}f = B
\ee
we can construct the solution to this equation as
\be
f = f_0 + \sum_i c_i f_i
\ee
Where the specific solution ($f_0$) is known as a \textbf{Particular Solution} and the $f_i$ solutions refer to the solutions of the associated homogeneous equation.

\subsection*{ODE: Linear, Homogeneous, Constant Coefficients}
We will start with the simplest ODEs, the linear, homogeneous, constant coefficient expressions.
Various books use different notation for presenting ODEs.
For example, it is very common in physics to consider functions that only depend on time ==$>$ x = x(t).

We can also define notation as
\be
x^n \equiv \frac{d^n}{dt}x \equiv D^n x
\ee
Hopefully it is clear that D is the linear differential operator.
If you apply a linear operator n times it is still linear therefore D$^n$ is also a linear operator.

The general form of these equations is
\be \label{eq:lin_cc}
\sum_{n=0}^N a_nx^{(n)} = f(t)
\ee

If we consider the homogeneous linear constant coefficients, than equation \ref{eq:lin_cc} simplifies to
\be \label{eq:lin_cc_homo}
\sum_{n=0}^N a_nx^{(n)} = 0
\ee
To solve these equations, we assume a solution of the form: x(t) = e$^{\lambda t}$ and substitute this into equation \ref{eq:lin_cc_homo}.
\be
\begin{split}
    \sum_{n=0}^N a_nx^{(n)} &= 0\\
    \sum_{n=0}^N a_n \left(e^{\lambda t}\right) ^{(n)} &= 0\\
    \sum_{n=0}^N a_n \lambda^n e^{\lambda t} &= 0\\
    e^{\lambda t} \sum_{n=0}^N a_n \lambda^n &= 0\\
    \sum_{n=0}^N a_n \lambda^n &= 0\\
\end{split}
\ee

The last line is known as the \textbf{Characteristic Polynomial} and it helps us find the general solution to the linear constant coefficient ODE.
The above polynomial is of degree N, and a theorem in  ODEs says that there exists exactly N roots (in general complex) that can potentially be degenerate.
%==============================================================================%
% Someone go find this theorem
%==============================================================================%

The reason for using this approach comes solely from mathematical intuition.
Our differential equation requires a function $x(t)$ such that the function and its derivatives cancel out to satisfy the RHS.
The only function we know of that satisfies this condition is x(t) = e$^{\lambda t}$.
In order to account for different multiples of the function, we include the $\lambda$ in our assumed solution, since its derivative is equal to $\lambda$ e$^{\lambda t}$.
One would think that trigonometric functions and several derivatives of itself can satisfy this condition as well.
While this is sometimes true, it is not true generally.
Even if you consider a combination of sine and cosine, such a function would not always work.
Consider the example ODE below,
\be
\frac{dy}{dt} + y = 0 .
\ee
Suppose that we assume the solution to be x(t) = sin($\lambda$ t) + cos($\lambda$ t).
Plugging this and its derivative into the original equation yields
\be
\lambda cos(\lambda t) - \lambda sin(\lambda t) + sin(\lambda t) + cos(\lambda t) = 0 .
\ee
However, there are no values of $\lambda$ that could satisfy this equation.
Therefore, we are only left with generally assuming that the solution to linear constant coefficient ODEs is an exponential function to avoid situations similar to the above example.

We begin exploring situations of non-degenerate and degenerate roots of the characteristic equation by assuming the non-degenerate case (i.e. $\lambda_i \neq \lambda_j$).
Since each root of the polynomial gives us the $\lambda_n$ in the general solution, we can write the general solution to the ODE as
\be
x(t) = \sum_{n=1}^N c_n e^{\lambda_n t}
\ee
Finding the general solution for the degenerate case is not as straightforward because if we have two roots that are the same (e.g. $\lambda_1$ = $\lambda_2$), we would not be able to form 2 unique solutions from these roots.
We would only be able to form one unique solution.
However, there are multiple ways to show that for each repeated root, each solution will contain some multiple of the variable in question.
For example, if we had a 3rd order linear ODE with repeated roots, $\lambda_1$ = $\lambda_2$ = $\lambda_3$, the general solution would be
\be
x(t) = c_1 e^{\lambda_1 t} + c_2 t e^{\lambda_1 t} + c_3 t^2 e^{\lambda_1 t} .
\ee
There are 2 ways to show that this solution holds.
Consider the following second order ODE with constant coefficients
\be \label{eq:lin_cc_homo_reproot}
a\frac{d^2 y}{dt^2} + b\frac{d y}{dt} + cy = 0
\ee
If the above ODE has two repeated roots, $\lambda_1$ and $\lambda_2$, we can say that one unique solution to the ODE is
\be
y_1(t) = c_1 e^{\lambda_1 t}
\ee
We can consider evaluating the following limit to obtain the resulting second unique solution
\be
\lim_{\lambda_1\to\ \lambda_2} \frac{e^{\lambda_1 t} - e^{\lambda_2 t}}{\lambda_2 - \lambda_1} = t e^{\lambda_2 t} .
\ee
To show the second way, we again consider equation \ref{eq:lin_cc_homo_reproot}.
Since one unique solution is $c_1 e^{\lambda_1 t}$, we can replace the constant $c_1$ with a function $v(t)$ and try to determine the function $v(t)$ such that it is also a solution of equation \ref{eq:lin_cc_homo_reproot}.

So the linear constant coefficient ODE problem is equivalent to finding the roots of a polynomial, and related to the linear algebra problem (which we will see later in the course).
%==============================================================================%
% Someone can explain this better / elaborate if they want
%==============================================================================%

\subsubsection*{Damped Harmonic Oscillator}
Let's consider an example, the \textbf{Damped Harmonic Oscillator}, a very common physics example with analytic solutions.
\be
\frac{d^2}{dt^2} x + 2\gamma \frac{d}{dt} x + \omega^2 x = 0
\ee
We will see this equation more, but the $\gamma$ is interpreted as the exponential decay, and the $\omega$ term refers to the frequency in physics models.
If we assume ==$>$ x = e$^{\lambda t}$ we can find the characteristic equation
\be
\begin{split}
\lambda^2 + 2\gamma\lambda  + \omega^2 = 0\\
\lambda = -\gamma \pm \sqrt{^2 - \omega^2}
\end{split}
\ee
%==============================================================================%
% Someone do the work for the derivation, I am not sure if the last line is correct...
%==============================================================================%
Where we see the last line is the characteristic equation, and it is a quadratic expression.

There are 3 different regiemes to consider
\benum
\item $\omega > \gamma$, the damped solution.
\item $\omega < \gamma$, the overdamped solution.
\item $\omega = \gamma$, the degenerate solution ($\lambda_1 = \lambda_2$).
\eenum

%==============================================================================%
% I AM HERE
%==============================================================================%





























\end{document}
