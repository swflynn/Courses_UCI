\documentclass{article}
%==============================================================================%
%	                          Packages                                     %
%==============================================================================%
% Packages
\usepackage[utf8]{inputenc}
\usepackage{graphicx}
\usepackage{amsmath}
\usepackage{braket}
\usepackage[margin=0.7in]{geometry}
\usepackage[version=4]{mhchem}
%==============================================================================%
%                           User-Defined Commands                              %
%==============================================================================%
% User-Defined Commands
\newcommand{\be}{\begin{equation}}
\newcommand{\ee}{\end{equation}}
\newcommand{\benum}{\begin{enumerate}}
\newcommand{\eenum}{\end{enumerate}}
\newcommand{\pd}{\partial}
\newcommand{\dg}{\dagger}
%==============================================================================%
%                             Title Information                                %
%==============================================================================%
\title{Chem237: Lecture 1}
\date{4/3/18}
\author{Shane Flynn}
%==============================================================================%
%	Everyone Please Make Comments if Something Needs to be Reviewed        %
%                           Or just fix it yourself!                           %
%==============================================================================%
\begin{document}
\maketitle
\section*{Course Overview}
The course will not directly follow the textbook, however, it will provide a rough guideline for important topics.
The lecture and homework are the most important aspects of the course.
The homework will be split into both analytical and numerical problems, any programming language is fine for the numerical problems, however, mathematica and other highly developed languages are discouraged.
For the numerical problems if you are asked to make an algorithm you CANNOT use pre-built algorithms available to the languages.

\section*{Chapter 1: Differential Equations}
A Differential Equation simply refers to an equation containing derivitives.
An \textbf{Ordinary Differential Equation} (ODE) refers to any differential equation with derivitives taken wrt a single variable.
In contrast a \textbf{Partial Differential Equation} (PDE) contains derivitives wrt more than 1 variable. 
We can further categorize differential equations (both ODE and PDE) into linear or nonlinear. 
As you can imagine linear equations are simplier to solve compared to nonlinear and ODEs are easier to solve than PDEs. 

\subsection*{Linear Equations}
Linear Equations are a general topic, there are linear differential equations, linear algebraic equations, and etc. 
A linear equation has the form
\be
\hat{A} f(x,y,\hdots) = B(x,y,\hdots)
\ee
Where the RHS (B) is known, and the function f is an unknown function. 
To be a linear equation, the operator $\hat{A}$ must be a linear operator.
For example take the equation above, and let $\hat{A}$ be the differential operator wrt a single dimension, and you get a linear ordinary differential equation. 

If the RHS (B) is equal to 0, the linear equation is called \textbf{Homogeneous} and if the RHs is anything other than 0 if it termed \textbf{Inhomogeneous}.

\subsubsection*{Homogeneous Linear Equations}
Consider equation \ref{eq:lin_homo} below
\be \label{eq:lin_homo}
\hat{A}f = 0
\ee
If we assume f$_1$ and f$_2$ are solutions to equation \ref{eq:lin_homo} than we immediately known another solution given by
\be
f_3 = c_1f_1 + c_2f_2
\ee
This result is true for any set of linear homogeneous equations. 
The linear combination of two different solutions to a linear homogeneous equation is another unique solution. 

\subsubsection*{Inhomogeneous Linear Equations}
If we now consider the Inhomogeneous linear equation, equation \ref{eq:lin_inhomo}
\be \label{eq:lin_inhomo}
\hat{A}f = B
\ee
If we know a specific solution to equation \ref{eq:lin_inhomo} than we can construct another solution as
\be
f = f_0 + \sum_i c_i f_i
\ee
Where the specific solution (f$_0$) is known as a \textbf{Particular Solution} and the f$_i$ solutions refer to the solutions of the associated homogeneous equation. 

\subsection*{ODE: Linear, Homogeneous, Constant Coefficients}
We will start with the simpliest ODEs, the linear, homogeneous, constnat coefficient expressions. 
Various books use different notation for presenting ODEs.
For example, it is very common in physics to considers functions that only depend on time ==$>$ x = x(t). 

We can also define notation 
\be
x^n \equiv \frac{d^n}{dt}x \equiv D^n x
\ee
Hopefully it is clear that D is the differential operator, it is a linear operator.
If you apply a linear operator n times it is still linear therefore D$^n$ is also a linear operator. 
The general form of these equations are
\be \label{eq:lin_cc}
\sum_{n=0}^N a_nx^{(n)} = f(t)
\ee

If we consider the homogeneous linear constant coefficients, than equation \ref{eq:lin_cc} simplifies to
\be \label{eq:lin_cc_homo}
\sum_{n=0}^N a_nx^{(n)} = 0
\ee
%==============================================================================%
% Someone Give motivation/intuition for this equation solution
%==============================================================================%
To solve we assume a solution of the form: x(t) = e$^{\lambda t}$ and substitute this into equation \ref{eq:lin_cc_homo}.
\be
\begin{split}
    \sum_{n=0}^N a_nx^{(n)} &= 0\\
    \sum_{n=0}^N a_n \left(e^{\lambda t}\right) ^{(n)} &= 0\\
    \sum_{n=0}^N a_n \lambda^n e^{\lambda t} &= 0\\
    e^{\lambda t} \sum_{n=0}^N a_n \lambda^n &= 0\\
    \sum_{n=0}^N a_n \lambda^n &= 0\\
\end{split}
\ee
The last line is known as the \textbf{Characteristic Polynomial} and is the general solution to the linear constant coefficient ODE.
The above polynimial is of degree N, and a theorem in  ODEs says that there exists exactly N roots (in general complex) that can potentially be degenerate. 
%==============================================================================%
% Someone go find this theorem
%==============================================================================%

Let's begin exploring these equations by assuming the non-degenerate case (i.e. $\lambda_i \neq \lambda_j$). 
This means we have N solutions, but any linear combination of these solutions also presents a new solution.
Therefore a general solution to the non-degenerate case is 
\be
x(t) = \sum_{n=1}^N c_n e^{\lambda_nt}
\ee
So the linear constant coefficient ODE problem is equivalent to finding the roots of a polynomial, and related to the linear algebra problem (which we will see later in the course). 
%==============================================================================%
% Someone can explain this better / elaborate if they want
%==============================================================================%

\subsubsection*{Damper Harmonic Oscillator}
Let's consider an example, the \textbf{Damped Harmonic Oscillator}, a very common physics example with analytic solutions. 
\be
\frac{d^2}{dt^2} x + 2\gamma \frac{d}{dt} x + \omega^2 x = 0
\ee
We will see this equation more, but the $\gamma$ is interpreted as the expotential decay, and the $\omega$ term refers to the frequency in physics models. 
If we assume ==$>$ x = x$^{\lambda t}$ we can find the characteristic equation
\be
\begin{split}
\lambda^2 + 2\gamma\lambda  + \omega^2 &= 0\\
-\gamma & \pm \sqrt{^2 - \omega^2}
\end{split}
\ee
%==============================================================================%
% Someone do the work for the derivation, I am not sure if the last line is correct...
%==============================================================================%
Where we see the last line is the characteristic equation, and it is a quadratic expression. 

There are 3 different regiemes to consider
\benum
\item $\omega > \gamma$, the damped solution.
\item $\omega < \gamma$, the overdamped solution.
\item $\omega = \gamma$, the degenerate solution ($\lambda_1 = \lambda_2$).
\eenum

%==============================================================================%
% I AM HERE 
%==============================================================================%

























\end{document}
