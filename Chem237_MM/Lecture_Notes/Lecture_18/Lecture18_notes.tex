\documentclass{article}
%=============================================================================80
%	                          Packages                                     %
%==============================================================================%
% Packages
\usepackage[utf8]{inputenc}
\usepackage{graphicx}
\usepackage{amsmath}
\usepackage{amssymb}
\usepackage{braket}
\usepackage{float}
\usepackage{subcaption}
\usepackage{nicefrac}
\usepackage[margin=0.7in]{geometry}
\usepackage[version=4]{mhchem}
%==============================================================================%
%                           User-Defined Commands                              %
%==============================================================================%
% User-Defined Commands
\newcommand{\be}{\begin{equation}}
\newcommand{\ee}{\end{equation}}
\newcommand{\benum}{\begin{enumerate}}
\newcommand{\eenum}{\end{enumerate}}
\newcommand{\pd}{\partial}
\newcommand{\dg}{\dagger}
\newcommand{\half}{\frac{1}{2}}
\newcommand{\nhalf}{\nicefrac{1}{2}}
\newcommand{\prt}{\frac{\pd}{\pd t}}
\newcommand{\prts}{\frac{\pd^2}{\pd t^2}}
\newcommand{\prx}{\frac{\pd}{\pd x}}
\newcommand{\prxs}{\frac{\pd^2}{\pd x^2}}
\newcommand{\bM}{\mbox{\bf M}}
\newcommand{\bS}{\mbox{\bf S}}
\newcommand{\bK}{\mbox{\bf K}}
\newcommand{\br}{\mbox{\bf r}}
%==============================================================================%
%                             Title Information                                %
%==============================================================================%
\title{Chem237: Lecture 18}
\date{5/16/18}
\author{Shane Flynn}
%==============================================================================%
\begin{document}

\maketitle

\section{Note}
I have just transcribed the lecture notes, no thought has gone in yet. We need to sit down and analyze this. 9-17-19

\section{PDEs}
Partial Differential Equations (PDEs) can have $\infty$ dimensionality, the problems usually involve boundry conditions with an arbitrary function. 
So you usually need to know the funciton and its derivative, and the value of the funciton and the vaue of the functions derivative at the boundry to solve. 

Consider a second order PDE (the wave equation with constant coefficient c)
\be
\prts U - c^2 \prxs U = 0 
\ee

We can give some initial conditions like 
\be
\begin{split}
    U(x,t=0) &= \phi x\\
    \prt U (x,t=0) &= \psi(x)
\end{split}
\ee

This PDE is linear and seperable, it is the easiest case to solve.
As we will see, given a certain set of boundry conditions we may not actually get a yseful solution (for example: could get a solution being an infinite sum we can't solve). 

One way to approach this problem is to guess a solution (standard approach in PDEs). 
If we assume the 
\be
U(x,t)=\left[f(x-ct)+g(x+ct)\right]
\ee

We need to satify the initial conditions
\be
\prx U = -cf'(x)+cg'(x) = \psi(x)
\ee

\be
\int dx \frac{dU}{dx} \Rightarrow f(x) - g(x) = \frac{-1}{c} \int^x dy\;\psi(y)
\ee
Where the lower boundis undefined, because f(x) and g(x) are defined up to a constant we do not know the lower bound. 

So second order PDE, infinte solutioon space, need to specify boundry conditions. 
\be\begin{split}
    \int f(x) - g(x) &=-\half\int^x dy\;\psi(y) \\
    -cf'(x) + cg'(x) &= \psi(x)
\end{split}
\ee
Solving these two equations simultansously gives the general solution to the 2D problem. 
\be
U(x,t) = \half\left[\phi(x-ct)+\phi(x+ct)\right] + \frac{1}{2c} \int_{x+t}^{x+ct} dy \; \psi(y) 
\ee

This can be easily generalized to the 3D case, U(x,y,z,t)
\be
\prts U - c^2 \Delta U = 0
\ee

%Guess we need to work out and show the 3D case

\section{Diffusion Equation}
Fick's law is a well known linear response expression of teh form
\be
J = -\frac{\sigma^2}{2}\prx \rho
\ee
Here we will derive the diffusion equaiton

COnsider an interval J(x) to J(x + $\Delta$x).
Assuming $\rho$=constant within our interval $\Delta$x then
\be
\prx \rho\; \Delta x = J(x) - J(x+\Delta x)
\ee

\be
\prt \rho = \frac{J(x) - J(x+\Delta x)}{\Delta x} \xrightarrow{\text{$\Delta$x=0}} \prt \rho= -\prx J
\ee

\be
\begin{split}
    J = -\frac{\sigma^2}{2}\prx \rho, \quad&  \prt \rho= -\prx J \Rightarrow \\
    \prt \rho &= \frac{\sigma^2}{2}\prxs \rho
\end{split}
\ee

\subsection{Solving This Equaiton}

\end{document}
