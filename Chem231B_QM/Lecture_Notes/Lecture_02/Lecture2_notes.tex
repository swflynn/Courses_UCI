\documentclass{article}
% Packages
\usepackage[utf8]{inputenc}
\usepackage{graphicx}
\usepackage{amsmath}
\usepackage{braket}
\usepackage[margin=0.7in]{geometry}
\usepackage{hyperref}
\usepackage[version=4]{mhchem}
% User-Defined Commands
\newcommand{\be}{\begin{equation}}
\newcommand{\ee}{\end{equation}}
\newcommand{\benum}{\begin{enumerate}}
\newcommand{\eenum}{\end{enumerate}}
\newcommand{\pd}{\partial}
\newcommand{\dg}{\dagger}
% Title Information
\title{Chem231B: Lecture 2}
\author{Shane Flynn}
\date{1/9/18}

\begin{document}
\maketitle

\section*{Lecture 1 Recap}
From Lecture 1 we found the eigenvalues (of $\hat{H}$) for the Quantum Harmonic Oscillator are given by
\be
E_n = \left( n+\frac{1}{2} \right) \hbar \omega
\ee
We also found that we could express the Hamiltonian in terms of our new creationa nd anniliation operators as 
\be
\hat{H} = \left(a^\dg\right) 
\ee
Finally we introduced (defined) the Number Operator $\hat{N}$ (which is a Hermitian Operator). 
\be
\hat{N} \equiv a^\dg a
\ee

\section*{Operator Algebra}
A cornerstone of this course will be the use of operators to solve Quantum Mechanics Problems. 
By using the operator formulation of Qm we can completly ignore the wavefunction and instead compute observables from just the algebraic manipulations of opertors.
This means we can ignore solving the various integrals and differential equations commonly found in the wave function formulation. 

\section*{The Ground State}
We can consider the ground state energy of the HO, which has an Eigenvalue (Eigen Energy) of
\be
E_0 = \frac{\hbar \omega}{2}
\ee
And its associated eigenvector satifies
\be
a\ket{\phi_0} = 0
\ee
This is a differential equation, we can substitute in the forms of the operators and find
\be
\begin{split}
    a = \frac{1}{\sqrt{2}}\left(\hat{X} + i\hat{P}\right)
\end{split}
\ee



\end{document}
